\documentclass{article}


% This is now the recommended way for checking for PDFLaTeX:
\usepackage{ifpdf}

% Use utf-8 encoding for foreign characters
\usepackage[utf8]{inputenc}

% Swedish grammar
\usepackage[swedish]{babel}

\usepackage[a4paper]{geometry}

% Space between paragraphs instead of indentation.
\usepackage{parskip}

\usepackage{listings}

\usepackage{fancyvrb}
\DefineShortVerb{\|}

\ifpdf
\usepackage[pdftex]{graphicx}
\else
\usepackage{graphicx}
\fi

\ifpdf
\DeclareGraphicsExtensions{.pdf, .jpg, .tif, .png}
\else
\DeclareGraphicsExtensions{.eps, .jpg}
\fi

\usepackage{float}

\pdfpxdimen=1in
\divide\pdfpxdimen by 300

\title{
  Projekt iMalloc \\
  Koddokumentation på gränssnittsnivå
}
\author{
  Niclas Edenvin \\
  Åke Lagercrantz \\
  Andreas Lelli \\
  Daniel Lindgren \\
  Elias Lundeqvist \\
  Jakob Sennerby
}

\date{2012-11-09}



\begin{document}

\maketitle

\newpage

\section{iMalloc}

\subsection*{Synopsis}
\begin{verbatim}
#include imalloc.h
struct style *iMalloc(chunk_size memsiz, unsigned int flags);
\end{verbatim}

\subsection*{Description}

iMalloc returns a pointer to a struct with memsiz reserved memory. 
iMalloc behaves differently depending on which flags has been chosen, 
the flags chose which functions to call.

memsiz is the user defined memory size and the flags changes
the way the memory is behaving and which functions to use.

\subsubsection*{memsiz} The number of bytes you want to reserve which can be entered in several forms.

|1Mb| - Reserves 1 megabyte of memory \\*
|1Kb| - Reserves 1 kilobyte of memory \\*
|10| - Reserves 10 bytes of memory \\*
|sizeof(int)*10| - Reserves enough memory to store 10 integers
\subsubsection*{flags}
Flags are entered separated by a plus sign. Eg. ASCENDING\_SIZE+GCD
The possible flags to choose from is listed below:


{\bf First} Choose how the freelist should be sorted

|ASCENDING_SIZE| - Sort the list with small objects first, large objects in the end \\*
|DESCENDING_SIZE| - Large objects first, small objects in the end \\*
|ADDRESS| - Sort the list depending on their adress, low adresses first, higher towards the end


{\bf Second} Choose which kind of memory manager to use (Note: only REFCOUNT and GCD can be combined)

|MANUAL| - Memory allocation using alloc and free \\*
|REFCOUNT| - Managed memory allocation using reference counter \\*
|GCD| - Managed memory allocation using the mark and sweep algorithm for garbage collection

Any other combinations will produce unspecified results and we cannot guaranty functionality
in those cases.

Usage examples:
Memory with a size of 2Mb, a freelist sorted after descending size and with garbage collection;
|iMalloc(2Mb, ASCENDING_SIZE+GCD)|

Memory with a size of |sizeOf(int)*10|, a freelist sorted after adress and refcount combined with GCD;
|iMalloc(sizeOf(int)*10, ADRESS+GCD+REFCOUNT)|


\section{Structs}
\subsection*{Synopsis}
\begin{verbatim}
typedef struct {
  TypedAllocator alloc;
  Global         collect;
} GC;
\end{verbatim}
\subsection*{Description}
alloc is a pointer to a function used to allocate memory within the adress space created by the iMalloc function.
collect is a pointer to a function used to start a garbage-collecting process according to the mark- and sweep algorithm.

\subsection*{Synopsis}
\begin{verbatim}
typedef struct {
  Local       retain;
  Manipulator release;
  Local       count;
} Refcount;
\end{verbatim}
\subsection*{Description}
This struct is used if a memory object is using refcount.
retain is used to increment the refcount
release is used to decrease the refcount
count returns the current refcount value

\subsection*{Synopsis}
\begin{verbatim}
typedef struct {
  RawAllocator alloc;
  Global       avail;
  Manipulator  free;
} manual, *Manual; 
\end{verbatim}
\subsection*{Description}
This struct is used if using manual memory manager.
alloc is a pointer to a function used to allocate memory within the adress space created by the iMalloc function.
avail returns the total size of the free space in the address space.
free frees an object in memory mem and returns the amount of memory freed.

\subsection*{Synopsis}
\begin{verbatim}
typedef struct {
  RawAllocator alloc;
  Refcount     rc;
  GC           gc;
} managed, *Managed;
\end{verbatim}
\subsection*{Description}
This struct is used if using managed memory manager.
alloc is a pointer to a function used to allocate memory within the adress space created by the iMalloc function.
rc is set when using refcount and/or gc is set when using refcount

\subsection*{Synopsis}
\begin{verbatim}
typedef union {
  manual  manual;
  managed managed;
} style;
\end{verbatim}
\subsection*{Description}
The style union is used as the return value from iMalloc(). This is to allow a return value of either a manual or managed memory manager, without using up unnecessary memory.
After calling iMalloc, you should immediately cast the return value to either a Manual or Managed pointer, like this.
\begin{verbatim}
Manual mem = (Manual) iMalloc(1 Mb, MANUAL + ASCENDING_SIZE);
\end{verbatim}

\section{Typedefs} 
\begin{verbatim}
typedef struct style *Memory;
typedef void *(*RawAllocator)(Memory mem, chunk_size size);
typedef void *(*TypedAllocator)(Memory mem, char* typeDesc);
typedef unsigned int(*Manipulator)(Memory mem, void *ptr);
typedef unsigned int(*Global)(Memory mem);
typedef unsigned int(*Local)(void *ptr);
\end{verbatim}

\end{document}