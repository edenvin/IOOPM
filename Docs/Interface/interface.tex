\documentclass{article}


% This is now the recommended way for checking for PDFLaTeX:
\usepackage{ifpdf}

% Use utf-8 encoding for foreign characters
\usepackage[utf8]{inputenc}

% Swedish grammar
\usepackage[swedish]{babel}

\usepackage[a4paper]{geometry}

% Space between paragraphs instead of indentation.
\usepackage{parskip}

\usepackage{listings}

\usepackage{fancyvrb}
\DefineShortVerb{\|}

\ifpdf
\usepackage[pdftex]{graphicx}
\else
\usepackage{graphicx}
\fi

\ifpdf
\DeclareGraphicsExtensions{.pdf, .jpg, .tif, .png}
\else
\DeclareGraphicsExtensions{.eps, .jpg}
\fi

\usepackage{float}

\pdfpxdimen=1in
\divide\pdfpxdimen by 300

\title{
  Projekt iMalloc \\
  Gränssnitten mellan modulerna
}
\author{
  Niclas Edenvin \\
  Åke Lagercrantz \\
  Andreas Lelli \\
  Daniel Lindgren \\
  Elias Lundeqvist \\
  Jakob Sennerby
}

\date{2012-11-09}



\begin{document}

\maketitle

\newpage

\section{Strukter som används i våra privata funktioner}
\begin{description} \parskip0pt
  \item[Lists] - Innehåller två pekare till chunks
    \begin{description} \parskip0pt
      \item[alloclist] - En lista av chunks som svarar mot objekt allokerade på heapen
      \item[freelist] - En lista med chunks som svarar mot det lediga utrymmet på heapen
    \end{description}

  \item[Chunk] - Innehåller pekarna start och next, storleken chunk_size samt mark-biten
    \begin{description} \parskip0pt
      \item[start] - Pekar till starten på objektet (som svarar mot chunken) som allokerats på heapen eller det fria utrymmet på heapen.
      \item[chunk_size] - Storleken på chunken
      \item[next] - Pekare till nästa chunk i listan.
      \item[mark-bit] - TRUE eller FALSE, beroende på om det finns ett program som använder chunken.
    \end{description}

  \item[priv_mem] - Innehåller addressrymden as, listorna lists samt funktionspekare
    \begin{description} \parskip0pt
      \item[AddressSpace] - En strukt som hör till rootset. Den håller koll på starten och slutet på det utrymme som allokerats med hjälp av imalloc.
      \item[lists]- Innehåller en pekare till strukten lists som i sin tur innehåller pekare till alloclistan och freelistan.
      \item[Funktionspekare] - 
    \end{description}

  \item[AddressSpace] - Innehåller två stycken char-pekare
    \begin{description} \parskip0pt
      \item[start] - En pekare till starten på det allokerade minnet.
      \item[end] - En pekare till slutet på det allokerade minnet.
    \end{description}

\end{description}

\section{imalloc}


\end{document}